The rapid development of Unmanned Aerial Vehicles (UAVs) has significantly impacted remote sensing, particularly in Land Use and Land Cover (LULC) classification. UAVs can capture high-resolution imagery, which is ideal for extracting fine-grained spatial details. However, challenges such as varying resolution, atmospheric distortion, and inconsistent lighting can affect the quality of the images. The application of deep learning techniques, particularly convolutional neural networks (CNNs), in enhancing UAV imagery via image upscaling is a promising direction, as seen in the proposed Gaurad-UAV model.

High-resolution data is crucial for accurate land use classification. Traditional satellite-based imagery often suffers from resolution constraints and longer revisit times. UAVs overcome these issues by offering greater spatial resolution and flexible data acquisition. Studies such as [6] and [7] have successfully demonstrated the use of UAV imagery for detailed classification of urban and agricultural landscapes. In [6], Bui et al. employed a CNN model combined with UAV images and digital surface models (DSMs) to achieve a high overall classification accuracy of over 91% in urban mapping tasks.
Deep learning models, especially transfer learning-based CNNs, have proven highly effective for LULC classification. Naushad et al. [7] used models like VGG16 and Wide Residual Networks (WRNs) to classify complex land use categories with an accuracy exceeding 98%, highlighting the capability of deep features to capture subtle variations in land cover types.
Image upscaling, or single-image super-resolution (SISR), has emerged as a key pre-processing step to improve classification performance. Tuna et al. [8] applied CNN-based super-resolution methods to remote sensing images, showing improved visual quality and feature detectability, which in turn enhanced classification outcomes. This process helps generate higher-resolution approximations of low-resolution images, mitigating the effects of noise and data sparsity in UAV captures.
The Gaurad-UAV framework is proposed as a comprehensive approach that leverages deep CNNs for image upscaling before land use classification. While specific architectural and experimental details are yet to be broadly published, the conceptual foundation aligns well with existing literature. By enhancing image resolution, Gaurad-UAV aims to improve the feature quality available to classification models, thereby increasing accuracy and robustness in heterogeneous environments.

In summary, the integration of UAV imagery with deep learning and image super-resolution techniques presents a strong foundation for enhanced LULC classification. Gaurad-UAV, as a deep learning-powered image enhancement pipeline, contributes meaningfully to this domain by addressing the challenge of limited image resolution in high-precision mapping applications.